\documentclass[10pt]{article}

% Preamble

\usepackage{amsmath,amsfonts,amssymb}
\usepackage[mathscr]{euscript}
%\usepackage[mathcal]{euscript}
\usepackage{mathrsfs}
\usepackage{graphicx}
\usepackage{float}
\usepackage{bbm}
\usepackage{braket}
\usepackage{tikz-feynman}
\usepackage{simpler-wick}
\usepackage{cancel}
\usepackage{stackengine}
\usepackage{slashed}
\usepackage{caption}
\usepackage{pgfplots}

\newcommand{\bigzero}{\mbox{\normalfont\Large\bfseries 0}}

\title{Lectures Notes For \\ An Introduction to Conformal Field Theory \\ A Course Given By Dr. Tobias Osborne} 
\author{Transcribed by Dr. Alexander V. St. John}

% The Document

\begin{document}

\maketitle

\clearpage

\section*{Lecture 1: Introduction to Conformal Field Theory}
\label{sec: lec1}

\noindent Recommended references:

\begin{itemize}
\item \textit{A Mathematical Introduction to Conformal Field Theory} by Schottenloher.
\item \textit{Applied Conformal Field Theory}, hep-th/9108028, by Ginsparg.
\item \textit{Conformal Field Theory} (``Yellow Book'') by Francesco, Mathieu, Senechal.
\end{itemize}

\noindent Why study conformal field theory (CFT)?

\begin{itemize}
\item CFT provides a good description of systems at or near criticality.
\item CFTs are the only true quantum field theories (QFTs), since they are cutoff-independent. One can think of QFTs as perturbations of CFTs. CFTs correspond to renormalization groups of fixed points, which dominate an effective theory at or near criticality.
\item CFTs can be made, by and large, mathematically rigorous, at least in $(1+1)$-dimensional theories. There are three major competing mathematical descriptions for CFT, and advances are being made towards a single, unifying description.
\end{itemize}

\noindent Prerequisites for this material:

\begin{itemize}
\item Advanced quantum mechanics
	\subitem E.g, many-body theory and Fock spaces.
\item Classical field theory
	\subitem E.g., symplectic geometry.
\item Quantum field theory.
\item Advanced quantum field theory.
\end{itemize}

\noindent What is CFT?

\begin{itemize}
\item A \textit{conformal field theory} is a field theory, quantum or classical, that is invariant, or symmetric, under a group of transformations called the \textit{conformal group} $G$.
\item In a classical field theory, this means that the equations of motion are left invariant.
\item In a quantum field theory, this means that, by Wigner's theorem, there is a projective unitary representation of the group $G$. In other words, symmetries, or transformations, that leave the transition amplitude invariant, are realized, up to a phase, by (anti)unitary operators.
\end{itemize}

\subsection*{Conformal Transformations in $d$ Dimensions}

\noindent Let $M = \mathbb{R}^{p,q}$ be a manifold $\mathbb{R}^d$, where $d = p+q$, and $p,q \in \mathbb{Z}_{\ge 0}$. To this manifold, assign the metric

\begin{equation}
g_{\mu\nu} \equiv \eta_{\mu\nu} = \text{diag} (1,1,\dots,1,-1,-1,\dots,-1)
\end{equation}

\noindent With the first $p$ entries equal to one, and the last $q$ entries equal to minus one. Note that this is not necessarily a Riemannian metric, since the signature can be negative. We have a few cases of interest for this metric

\begin{itemize}
\item $p=d$ , Riemannian.
\item $p = d-1$, $q=1$, Lorentz.
\item $q > 1$, e.g., $q=2$, AdS-CFT correspondence.
\end{itemize}

\noindent A conformal transformation leaves the metric invariant up to a scale factor. Consider a smooth change of coordinates

\begin{equation}
x \rightarrow x' = x' (x) \text{ , with } x = (x^1, x^2, \dots, x^p, x^{p+1}, \dots, x^{p+q})
\end{equation}

\noindent Such that the metric, a type-$(2,0)$ tensor, undergoes an \textit{active coordinate} transformation as

\begin{align}
g_{\mu\nu} (x) \rightarrow g'_{\mu\nu} (x') &= \frac{\partial x^\alpha}{\partial x'^\mu} \frac{\partial x^\beta}{\partial x'^\nu} g_{\alpha \beta} (x') \\ 
&= \Omega (x') g_{\mu\nu} (x').
\end{align}

\noindent Where $\Omega(x') > 0$ is the (local) scale factor. Note that if the scale factor is zero, then we have a singularity, which we will discuss later. Such a transformation is called \textit{conformal}, and these transformations preserve angles

\begin{equation}
\angle \theta = \frac{g_{\mu\nu} u^\mu v^\nu}{\sqrt{(g_{\mu\nu} u^\mu v^\nu)^2}}.
\end{equation}

\noindent The \textit{conformal group} of a manifold M is denoted by $\text{Conf}(M)$, and is the connected component of the group of all conformal transformations of $M$ containing the identity, in a compact, open topology. \\

\noindent So, in a quantum conformal field theory, we are looking for a Hilbert space $\mathcal{H}$ and a projective unitary representation of the group $G$ for \textit{local} QFTs

\begin{equation}
G \rightarrow \pi (G).
\end{equation}

\noindent This is unexpectedly difficult, and makes for a very rich field of study, since there is a tension between knowing the unitary representations of symmetries and demanding that the representation is locally implementable. \\

\noindent To classify the conformal group on our chosen manifold $G = \text{Conf} (\mathbb{R}^{p,q})$, consider an infinitesimal conformal (active coordinate) transformation on the spacetime coordinates

\begin{equation}
x^\mu \rightarrow x'^\mu = x^\mu +  \epsilon^\mu (x)
\end{equation}

\noindent Which also can act on the metric, since the conformal transformation is a more general, weaker constraint on the spacetime coordinates and the metric space. The conformal transformation places constraints on $\epsilon$, as (\textbf{Exercise})

\begin{equation}
g_{\mu\nu} \rightarrow g'_{\mu\nu} = g_{\mu\nu} + (\partial_\mu \epsilon_\nu + \partial_\nu \epsilon_\mu) + \mathcal{O} (\epsilon^2).
\end{equation}

\noindent To satisfy the constraint placed by conformal invariance on the metric (c.f., $g_{\mu\nu} \rightarrow g'_{\mu\nu} (x') = \Omega(x') g_{\mu\nu} (x')$), as well as the constraint that the conformally transformed metric is still proptional to the diagonal flat spacetime metric $g'_{\mu\nu} \propto \eta_{\mu\nu}$, we must have that the second term be diagonal, proportional to $\eta_{\mu\nu}$

\begin{align}
(\partial_\mu \epsilon_\nu + \partial_\nu \epsilon_\mu) &\propto \eta_{\mu\nu} \\
\implies (\partial_\mu \epsilon_\nu + \partial_\nu \epsilon_\mu) &= \text{constant} \cdot \eta_{\mu\nu}
\end{align}

\noindent Take the trace of each side and solve for the constant

\begin{equation}
\text{constant} = \frac{2 (\partial \cdot \epsilon)}{d}
\end{equation}

\noindent So, the conformal transformation on the metric reads, tossing out higher order terms,

\begin{equation}
g'_{\mu\nu} = g_{\mu\nu} + \frac{2 (\partial \cdot \epsilon)}{d} g_{\mu\nu}.
\end{equation}

\noindent And for the proportionality relation above, we have

\begin{equation}
(\partial_\mu \epsilon_\nu + \partial_\nu \epsilon_\mu) = \frac{2}{d} (\partial \cdot \epsilon) \eta_{\mu\nu}.
\end{equation}

\noindent Combining this with the conformal transformation of the metric and comparing to the metric transformation law, we get that the scale factor $\Omega(x)$ for the conformal transformation of the spacetime metric is 

\begin{equation}
\Omega(x) = 1+\frac{2}{d} (\partial \cdot \epsilon).
\end{equation}

\noindent Then, expand and equate mixed partial derivatives (to third order), and we get $d^2$ partial differential equations of the form (\textbf{Exercise})

\begin{equation}
(\eta_{\mu\nu} \Box + (d-2) \partial_\mu \partial_\nu) (\partial \cdot \epsilon) = 0
\end{equation}

\noindent Where $\Box = \eta^{\mu\nu} \partial_\mu \partial_\nu$ is the d'Alembertian operator.

\subsubsection*{Classification of Infinitesimal Conformal Translations for $d>2$}

\noindent Recall that an infinitesimal transformation can be made global via exponentiaion.

\noindent By combining the two equations

\begin{align}
(\partial_\mu \epsilon_\nu + \partial_\nu \epsilon_\mu) &= \frac{2}{d} (\partial \cdot \epsilon) \eta_{\mu\nu} \\
(\eta_{\mu\nu} \Box + (d-2) \partial_\mu \partial_\nu) (\partial \cdot \epsilon) &= 0
\end{align}

\noindent We find that third order derivatives of $\epsilon(x)$ vanish and $\epsilon(x)$ is at most quadratic. This leaves four types of infinitesimal transformations, via $\epsilon$, allowable in a conformal transformation: one constant, two linear, and one quadratic in spacetime coordinates.

\begin{enumerate}
\item Spacetime translations
	\subitem $\epsilon = a^\mu$.
\item Rotations
	\subitem  $\epsilon^\mu = \omega^\mu_{\,\,\nu} x^\nu$, $\omega$ antisymmetric.
\item Scale transformations
	\subitem $\epsilon^\mu = \lambda x^\mu$, $\lambda > 0$.
\item Special conformal transformations (SCT; inversion through a sphere)
	\subitem $\epsilon^\mu = b^\mu x^2 - 2 x^\mu (b \cdot x)$.
\end{enumerate}

\noindent Note that Lorentz and Poincar\'e transformations are always subgroups of the conformal group, leaving the metric invariant. since $\omega$ corresponds to boosts and Euclidean affine rotations complete the Poincar\'e group. \\

\noindent \textbf{Theorem} \\

\noindent Every conformal transformation that acts on an connected subset of Minkowski space, including the whole space itself, $\varphi : \, U \subset \mathbb{R}^{p,q}$, where $p+q > 2$, is a composition of 

\begin{itemize}
\item a translation
	\subitem $x^\mu \rightarrow x^\mu + a^\mu$, where $a \in \mathbb{R}^d$,
\item an orthogonal transformation (rotation)
	\subitem $x \rightarrow \Lambda x$, where $\Lambda \in O(p,q)$,
\item a dilation (scale)
	\subitem $x^\mu \rightarrow \lambda x^\mu$, where $\lambda \in \mathbb{R}^+$,
\item and an SCT
	\subitem $x \rightarrow \frac{x^\mu - b x^2 }{1-2b \cdot x + b^2 x^2}$, where $b \in \mathbb{R}^q$.
\end{itemize}

\noindent Note that it is possible to find a vector $b$ such that the denominator is equal to zero, the SCT is not invertible, and this is no longer a group. Also note that if we don't compactify the space and include $\infty$ as a point available to the conformal transformation, the group becomes significantly smaller and more constrained. \\

\subsubsection*{Classification of Infinitesimal Conformal Translations for $d=2$}

\noindent If $d=2$, the spacetime metric becomes the identity

\begin{equation}
g_{\mu\nu} = \delta_{\mu\nu}
\end{equation}

\noindent And $(\partial_\mu \epsilon_\nu + \partial_\nu \epsilon_\mu) = \frac{2}{d} (\partial \cdot \epsilon) \eta_{\mu\nu}$ becomes the Cauchy-Riemann equations and $\epsilon(x)$ is complex-valued, complex-differentiable, and analytic

\begin{equation}
\partial_1 \epsilon_1 = \partial_2 \epsilon_2 \text{ and } \partial_1 \epsilon_2 = - \partial_2 \epsilon_1.
\end{equation}

\noindent Introduce the complex coordinates

\begin{equation}
z = x^1 + i x^2 \text{ and } \bar{z} = x^1 - i x^2.
\end{equation}

\noindent Then we can complexify $\epsilon$ as

\begin{equation}
\epsilon(z) = \epsilon^1 + i \epsilon^2 \text{ and } \bar{\epsilon}({\bar{z}}) = \epsilon^1 - i \epsilon^2.
\end{equation}

\noindent Two-dimensional global (exponentiated infinitesimal) conformal transformations correspond to \textit{entire} (no singularities, all invertible), \textit{holomorphic} functions $z \rightarrow f(z)$ with holomorphic inverses $f^{-1} (z)$. The only allowable form for a conformal transformation that corresponds to an entire, holomorphic function is linear in the complex coordinates

\begin{equation}
f(z) = \alpha z + \beta, \text{ where } \alpha, \beta \in \mathbb{C}.
\end{equation}

\noindent We may expect a larger group of symmetries with entirety and holomorphism enforced, since the space seems less constrained, but this actually constrains the space more and the group becomes smaller. So, if we were to not compactify, and add infinity as a point, as we demonstrated, the conformal space becomes linear and boring: only rotations and scaling are allowed.\\

\noindent To include this complex representation of the spacetime coordinates, we \textit{extend} our manifold to the complex numbers $\mathbb{C}$ and compactify complex space to a Riemann sphere $\mathbb{C} \cup \{ \infty \}$ we get the proper space for the conformal transformations to act in

\begin{equation}
\mathbb{R}^{2,0} \rightarrow \mathbb{C} \rightarrow \mathbb{C} \cup \{\infty\} \rightarrow \text{Conf}(\mathbb{C} \cup \{\infty\} )
\end{equation}

\begin{figure}[H]
	\centering
	\includegraphics[width=3in]{images/riemann_sphere.png} 
\end{figure} 

\noindent Where the conformal group is

\begin{equation}
\text{Conf}(\mathbb{C} \cup \{ \infty \} ) = \Big{\{} f(z) = \frac{\alpha z + \beta}{\gamma z + \delta}; \, \alpha, \beta, \gamma, \delta \in \mathbb{C}, \alpha\delta - \beta \gamma \ne 0 \Big{\}}.
\end{equation}

\noindent This is also called the group of Moebius transformations, and is a slightly larger group of conformal transformations (symmetires), since we can map to and from infinity as a point. \\

\subsubsection*{Summary}

\noindent A conformal field theory is a local quantum field theory that is invariant under the conformal group, a set of transformations, a change in coordinates, that leave the metric invariant up to a scale factor. In different spacetime dimensions, the conformal group takes on significantly different forms. \\

\noindent The global conformal group in dimensions greater than two is comprised of translations, rotations, scaling, and special conformal transformations, as well as dimensions equal to two, as long as the space is compactified. If singularities are included, functions with poles are allowed, the symmetry gets larger.

\clearpage

\section*{Lecture 2: Local Conformal Transformations}
\label{sec: lec2}

\input{chapters/lec2.tex}

\clearpage

\section*{Lecture 3: Classical Conformal Field Theory}
\label{sec: lec3}


\noindent We continue our discussion of systems that exhibit conformal symmetries. These symmetries are contained in the conformal group called $\text{Conf}(\mathbb{R}^{p,q})$, which is the connected component containing the identity of all conformal diffeomorphisms of the pseudo-Riemannian manifold $\mathbb{R}^{p,q}$. \\

\noindent We discussed the infinitesimal conformal transformation, which led to a Lie algebra, the Witt algebra, in $(1+1)$ and $(2,0)$ dimensions. For $d = p+q = 2$, there is a bigger symmetry group (less constrained), yielding more conserved quantiteis, more degrees of freedom of the system. If $d \ne 2$, the symmetry group is too constrained to be that interesting. \\

\noindent A \textit{conformal theory} is a theory with a representation of the group $G = \text{Conf}(\mathbb{R}^{p,q})$. This group contains transformations corresonding to temporal translations, spatial translations, boosts, dilations, and special conformal transformations (inversion about the origin, translation, and a second inversion about the origin). So, a conformal theory has a Hamiltonian $H$ built in, since it is the generator of time translations. \\

\noindent Note that in a nonrelativistic theory, we demand that the Hamiltonian $H$ commutes with everything, which introduces symmetries of the system, but the inclusion of boosts requires a relativistically invariant theory. This constrains the theory further to allow only certain symmetries and exhibit the desired properties. \\

\noindent Note that the Lorentz boost mixes energy and momentum through conjugation of spatial translations to temporal translations. This conjugation requires that all types of possible transformations in a nonrelativistic theory must be represented all at once, and they are not independent of each other. \\

\noindent Another property we need for our theory is \textit{locality}. \\

\noindent So, we have a collection of observables $\phi_a (x)$, where $x \in \mathbb{R}^{p,q}$ and $a \in I$, an index set (labels by particle types, vector quantities, etc.), which can be classical (functions on phase space), quantum (self-adjoint operators), or even probabilistic (element of ordered unit vector space). \\

\noindent A representation of a group of symmetries is a map $\pi$ that can be

\begin{align}
\text{finite } &\pi: \, G \rightarrow M_n (\mathbb{C}) &&\text{, the } n \times n \text{ matrices over the complex numbers} \\
\text{infinite } &\pi : \, G \rightarrow \mathcal{B}(\mathcal{H}) &&\text{, the bounded operators on a Hilbert space, for example.}
\end{align}

\noindent The concern is that a given representation does not necessarily yield a set of observables $\phi_a (x)$. In the event that it does, it is likely that a representation which furnishes a collection of (local) observables is \textit{reducible}, and can be decomposed into a direct sum of \textit{irreducible} representations, or \textit{irreps}. This makes for an infinite number of ways to build reducible representations. \\

\noindent So, although we can write down an irreducible representation of $G$ and attempt to enforce locality, we prefer to take the stance, and shall from this point on, that the locality of the theory is the most important property, and find irreducible representations from there. \\

\subsection*{Classical field representations of conformal symmetries}

\noindent The concept of the field easily puts forth the idea of locality, but what constraints does conformal symmetry place on a classical field? \\

\noindent Recall for symmetries in a classical field theory start with the action

\begin{equation}
S = \int d^d x \, \mathcal{L} (\phi, \partial_\mu \phi), \text{ where } \phi = \{ \phi_a (x) \}.
\end{equation}

\noindent By writing down the action, we have assumed $(1)$ that the equations of motion are represented by an action and $(2)$ that the Lagrangian density depends only on the field and its first derivatives. We have effectively thrown out all non-local theories. \\

\noindent So, a symmetry transformation takes a spacetime location and maps it to its image under that transformation: $x \rightarrow x'$. If the transformation is \textit{active}, the then fields transform as well

\begin{equation}
\phi (x) \rightarrow \phi' (x') \equiv \mathcal{F}(\phi(x))
\end{equation}

\noindent Where we note that $\mathcal{F}(\phi(x))$ depends on the previous field configuration. \\

\noindent For example, in an active rotation of a vector field $\mathbb{R}^{2,0}$, a nontrivial representation rotates the spacetime coordinate as well as the vector at each spacetime coordinate, the field (A trivial representation will not rotate the vector.)

\begin{figure}[H]
	\centering
	\includegraphics[width=4in]{images/active_rotation.png} 
\end{figure} 

\noindent After the rotation, the new field configuration at $x$ is

\begin{equation}
\phi'_a (x) = \sum_b \pi (\mathcal{O})_{ab} \phi_b (\mathcal{O}^{-1} x).
\end{equation}

\noindent The trivial representation of the field component $b$ would simply be the identity $\pi(\mathcal{O})_{ab} = \delta_{ab}$, and the fundamental, nontrivial representation is written

\begin{equation}
\pi (\mathcal{O})_{ab} = [ \mathcal{O} ] _{ab}.
\end{equation}

\noindent How does the action $S$ transform under a symmetry transformation?

\begin{equation}
S' = \int d^d x \, \Big{|} \text{det} \left( \frac{\partial x'^\mu}{\partial x^\nu} \right) \Big{|} \mathcal{L} \left( \mathcal{F} \left( \phi (x) \right), \frac{\partial x^\nu}{\partial x'^{\mu}} \partial_\nu \mathcal{F} \left( \phi (x) \right) \right)
\end{equation}

\noindent We also know that our theory is a conformal field theory, conformally invariant, conformally symemtric if the equations of motion are invariant. This is equivalent to the Lagrangian density transforming up to a total derivative

\begin{equation}
\mathcal{L}' = \mathcal{L} + \text{total derivative}.
\end{equation}

\noindent Now, let's study the infinitesimal generators of the conformal group $\text{Conf} ( \mathbb{R}^{p,q} )$.

\begin{itemize}
\item Translation
	\subitem $P_\mu = -i \partial_\mu$
\item Dilation
	\subitem $D = -i x^\mu \partial_\mu$
\item Rotation (Boost)
	\subitem $L_{\mu\nu} = i (x_\mu \partial_\nu - x_\nu \partial_\mu)$
\item Special Conformal
	\subitem $K_\mu = -i (2 x_\mu x'^\nu \partial_\nu - x^2 \partial_\mu)$
\end{itemize}

\noindent Work out the commutation relations to form a Lie algebra (\textbf{Exercise}).

\begin{align}
[D, P_\mu] &= i P_\mu \\
[D, K_\mu] &= -i K_\mu \\
[K_\mu, P_\nu] &= 2i (\eta_{\mu\nu} D - L_{\mu\nu} ) \\
[K_\rho, L_{\mu\nu}] &= i (\eta_{\rho\mu} K_\nu - \eta_{\rho\nu} K_\mu) \\
[P_\rho, L_{\mu\nu}] &= i (\eta_{\rho\mu} P_\nu - \eta_{\rho\nu} P_\mu ) \\
[L_{\mu\nu}, L_{\rho\sigma}] &= i (\eta_{\nu\rho} L_{\mu\sigma} + \eta_{\mu\sigma} L_{\nu\rho} - \eta_{\mu\rho} L_{\nu\sigma} - \eta_{\nu\sigma} L_{\mu\rho} )
\end{align}

\noindent And the rest commute. \\

\noindent Our task now is to find out which kinds of fields transform under the conformal group and give representations of the conformal group. We already know that a field transforming under $\text{Conf}(\mathbb{R}^{p,q})$, a general conformal transformation, transforms under the subgroup Poincar\'e, which is generated by translation $P_\mu$ and rotation $L_{\mu\nu}$, as

\begin{equation}
\phi_a (x) \rightarrow_{\text{Poincar\'e}} \sum_b \pi(\Lambda)_{ab} \phi_b (\Lambda^{-1}x).
\end{equation}

\noindent Let's focus on the subgroup of $\text{Conf}(\mathbb{R}^{p,q})$ which leave the orgin fixed: rotations, dilations, and SCTs. The infinitesimal generators of this subgroup form a subalgebra by exponentiation

\begin{equation}
\Lambda = e^{i \omega^\alpha G_\alpha}, \,\, \text{where } \omega^\alpha \text{ is infinitesimal.}
\end{equation}

\noindent The group element $G_\alpha$ can be $K_\mu$, $D$, or $L_{\mu\nu}$. At the origin, the Poincar\'e transformation  of the field looks like

\begin{equation}
\phi_a (x=0) \rightarrow \sum_b \pi (e^{i \omega^\alpha G_\alpha})_{ab} \phi_b (\Lambda^{-1} x = 0)
\end{equation}

\noindent Where the representation is, by Taylor expansion,

\begin{equation}
\pi(e^{i \omega^\alpha G_\alpha}) = \pi(\mathbb{I}) + i \omega^\alpha \pi(G_\alpha).
\end{equation}

\noindent Rename the representations of the generators (group elements)

\begin{align}
\pi(D) &= \tilde{\Delta} (\text{scaling dimension})\\
\pi(K_\mu) &= \kappa_\mu \\
\pi(L_{\mu\nu}) &= S_{\mu\nu} \, (\text{spin}).
\end{align}

\noindent The commutation of these representations are then

\begin{align}
[ \tilde{\Delta}, S_{\mu\nu} ] &= 0 \\
[ \tilde{\Delta}, \kappa_\mu ] &= -i \kappa_\mu \\
[ \kappa_\mu, \kappa_\nu ] &= 0.
\end{align}

\noindent Now, suppose that the generators $S_{\mu\nu}$ are irreducible representations, irreps, of the Lorentz group, the group that describes spin/helicity. By Schur's lemma (\textbf{Exercise}), we find that the scaling dimension is trivial, and, in turn, by the commuation relations, that all generators $\kappa_\mu$ are also trivial

\begin{equation}
\tilde{\Delta} \propto \mathbb{I} \implies -i \kappa_\mu = 0.
\end{equation}

\noindent Now use this fact to show how dilations act on the fields. The coordinates transform as

\begin{align}
x &\rightarrow \lambda x \\
x &\rightarrow \lambda^\epsilon \lambda^\epsilon \dots \lambda^\epsilon x
\end{align}

\noindent At the origin, the field transforms as

\begin{align}
\phi_a (x=0) \rightarrow&  (\mathbb{I} + i \epsilon \tilde{\Delta}) \dots (\mathbb{I} + i \epsilon \tilde{\Delta}) \phi_a (0) \\
&= \lambda^{i \tilde{\Delta}_a} \phi_a (0) \\
&= \lambda^{-\Delta_a} \phi_a (0)
\end{align}


\noindent Where we used the Taylor expansion of the infinitesimal $\epsilon$, and the last line uses $\tilde{\Delta} = i \Delta \mathbb{I}$, since Schur's lemma tells us that the scaling deimension is trivial. \\

\noindent So, every conformal field has a behavior under dilations, defined by the scaling dimension $\Delta_a$, with Jacobian

\begin{equation}
\Big{|} \frac{\partial x'}{\partial x} \Big{|} = \Lambda^{-\frac{d}{2}}, \text{ where } \Lambda = \lambda^{-2}.
\end{equation}

\noindent And the metric transforms under dilations as

\begin{equation}
g'_{\mu\nu} = \lambda^{-2} g_{\mu\nu}.
\end{equation}

\noindent Putting all this together, the field now transforms as 

\begin{equation}
\phi_a (x) \rightarrow \phi'_a (x') = \Big{|} \frac{\partial x'}{\partial x} \Big{|}^{-\frac{\Delta}{d}} \phi_a (x).
\end{equation}

\noindent Filling in the Jacobian, the new field in terms of the orignial field and the original spacetime location is

\begin{align}
\phi'_a (x) &= \sum_b \pi(\Lambda)_{ab} \phi_b (\Lambda^{-1} x) \\
&= \sum_b [ \lambda^{-\Delta}]_{ab} \phi_b (\Lambda^{-1} x) \\
&= \lambda^{-\Delta} \phi_a (\Lambda^{-1} x) \\
&= \lambda^{-\Delta} \phi_a (\lambda^{-1} x).
\end{align}

\noindent By the Baker-Campbell-Hausdorff (BCH) formula, we know how the new field looks at all spacetime lcoations, not just the origin (\textbf{Exercise})

\begin{align}
e^{i x^\rho P_\rho} D e^{-i x^\rho P_\rho} &= D + x^\nu P_\nu \\
&\implies D \phi_a (x) = (-i x^\nu \partial_\nu + \tilde{\Delta} ) \phi_a (x).
\end{align}

\noindent In the next lecture, we give this the quantum treatment, where we will look for unitary representations (self-adjoint operators), that are labelled by quantum numbers, such as spin, scaling dimension, and central charge. The central charge will require projective unitary representations.


\clearpage

\section*{Lecture 4: Constraints of Conformal Invariance on Quantum Field Theories}
\label{sec: lec4}


\noindent Recall that a scale transformation on a classical field is characterized per field, indexed by $a$, by the scaling dimension $\Delta_a$ in the mapping

\begin{equation}
\phi_a (x) \rightarrow \lambda^{-\Delta_a} \phi_a (x).
\end{equation}

\noindent We now make a move to study quantum fields, using the classical limit to check our results. So, suppose we have done all of the hard work of quantization and have a quantum field theory. \\

\noindent \textbf{Digression}: In $(1+1)$ dimensions, some approaches to quantization include

\begin{itemize}
\item Vertex operator algebras.
\item (Local) Algebraic QFT
	\subitem Top-down approach where one attaches an algebra of observables to each point in spacetime and then makes sense of it as a field theory.
\item Functors from $n$-Categories to Hilbert spaces: $n$Cat $\rightarrow$ Hilb (Segal).
\end{itemize}

\noindent  \textbf{Digression}: A main assumption of this course is that quantum field theories are a subset of quantum theories, although some schools argue that QFT is done via path integrals, which is not obviously a quantum theory. \\

\noindent The data of a quantum field theory includes

\begin{itemize}
\item Hilbert space of states $\mathcal{H}$
\item Projective unitary representation of the conformal group $U(g)$, where $g \in \text{Conf}(\mathbb{R}^{p,q})$
\item Vacuum (reference) state $\ket{0} \in \mathcal{H}$ \\
	\\ Invariant, up to a phase, under global symmetries, but not necessarily local symmetries: $U(g) \ket{0} = e^{i \varphi (g)} \ket{0}$. \\
	\\ If $p+q > 2$, then the global and local (full conformal) symmetries coincide. \\
	\\ If $p=q=1$, then the local conformal group of transformations (diffeomorphisms of the circle) is much larger than the global conformal group, making the global transformations subgroup of the local transformations. \\
\item Observables.
\end{itemize}

\noindent For observables, we demand that we can measure some set of local observables from the set of self-adjoint linear operators on the Hilbert space

\begin{equation}
A_{j,x} \in L_{\text{self-adjoint}} (\mathcal{H}); \,\, x \in \mathbb{R}^{p,q}; \,\, j \in \text{index set (e.g., particle type)}.
\end{equation}

\noindent \textbf{Digression}: We concern ourselves with \textit{local} observables, and non-local observables are difficult to imagine. An example of a non-local observable that shows up in gauge theory is the Wilson loop or Wilson line, since they are spread over a nonzero dimensional submanifold of the gauge theory's manifold. \\

\noindent Note that locality is induced by causality, such that if $x-y$ is spacelike, then $A_{j,x}$ and $A_{k,y}$ are jointly observable, $\forall \, j,k \text{ and } x,y$. \\

\subsection*{Quasi-Primary Observables}

\noindent The \textit{quasi-primary} of a field is a subset of local observables $\{ A_{j,x}: \, j \in J, x \in \mathbb{R}^{p,q} \}$ with the additional properties to satisfy the assumed constraints that enforce conformal invariance of the system, denoted here by $\{ \hat{\phi}_k (x): \, k \in K \}$ that transform as

\begin{equation}
U(g): \,\, \hat{\phi}_k (x) \rightarrow U^\dagger (g) \hat{\phi}_k (x) U(g) = \Big{|} \frac{\partial x'}{\partial x} \Big{|}^{\frac{\Delta_k}{d}} \hat{\phi}_k (x')
\end{equation}

\noindent Where $x' = gx$, and $g \in \text{Conf}(\mathbb{R}^{p,q})$ is a conformal transformation. \\

\noindent Now we must demonstrate that these assumptions yield nontrivial examples of fields. Rest assured that there are such fields, such as free bosons and free fermions. \\

\noindent By these assumptions, the $n$-point correlation function, which is observable via scattering experiments,  transforms under conformal transformations as

\begin{equation}
\bra{0} \hat{\phi}_{k_1} (x_1) \dots \hat{\phi}_{k_n} (x_n) \ket{0} = \Big{|} \frac{\partial x_1'}{\partial x_1} \Big{|}^{\frac{\Delta_{k_1}}{d}} \dots \Big{|} \frac{\partial x_n'}{\partial x_n} \Big{|}^{\frac{\Delta_{k_n}}{d}} \bra{0} \hat{\phi}_{k_1} (x'_1) \dots \hat{\phi}_{k_n} (x'_n) \ket{0}.
\end{equation}

\noindent This equation constrains the structure of the $n$-point correlation functions. To understand these constraints, let's analyze how each type of transformation of the conformal group constrains the invariants. \\

\noindent \textbf{Translations}: For $x_j, x_k \in \mathbb{R}^{p,q}$, $j, k = 1, \dots, n$, the difference $x_j - x_k$ is invariant, and there are $d(n-1)$ such quantities. \\

\noindent \textbf{Rotations}: For spinless objects (in large enough dimension $d$), the length $r_{jk} \equiv |x_j - x_k |$ is invariant, and there are ${n \choose 2}$ such quantities. \\

\noindent \textbf{Dilations}: Under the scale transformations, the length $r_{jk}$ is clearly not invariant, but the ratio $\frac{r_{jk}}{r_{lm}}$ can be invariant. \\

\noindent \textbf{SCTs}: Under special conformal transformations, invariant quantities must be cross ratios of the form $\frac{r_{jk} r_{lm}}{r_{jl} r_{km}}$, since the squared length under SCTs transforms as

\begin{equation}
|x'_1 - x'_2 |^2 = \frac{|x_1 - x_2 |^2}{(1 + 2 b \cdot x_1 + b^2 x_1^2) (1 + 2 b \cdot x_2 + b^2 x_2^2)}.
\end{equation}

\subsection*{Two-Point Correlation Functions}

\noindent Consider the classical two-point correlation function, or Green's function, of the quasi-primary fields

\begin{equation}
G^{(2)} (x_1, x_2) = \bra{0} \phi_1 (x_1) \phi_2 (x_2) \ket{0} = \Big{|} \frac{\partial x_1'}{\partial x_1} \Big{|}^{\frac{\Delta_{1}}{d}} \Big{|} \frac{\partial x_2'}{\partial x_2} \Big{|}^{\frac{\Delta_{2}}{d}} \bra{0} \hat{\phi}_{1} (x'_1) \hat{\phi}_{2} (x'_2) \ket{0}.
\end{equation}

For simplicity, let us denote 
\begin{equation}
	G^{(2)} (x_1, x_2) =f(x_1,x_2) 
\end{equation}

\noindent Exploiting the assumed conformal symmetries of the system, use the fact that the Jacobian for a translation, as well as for a rotation, is equal to one, such that $\Big{|} \frac{\partial x'}{\partial x} \Big{|} = 1$, and, therefore, the Green's function can only depend on the length 
\begin{equation}
	f(x_1,x_2) = f(Rx_1 + a, Rx_2 + a),
\end{equation}
requiring that
\begin{equation}
G^{(2)} (x_1, x_2) = f(|x_1 - x_2|) = f(r_{12}).
\end{equation}

\noindent For a dilation, the Jacobian obeys a scale factor, such that $\Big{|} \frac{\partial x'}{\partial x} \Big{|} = \lambda^{\Delta}$, and the Green's function has the form

\begin{equation}
G^{(2)} (x_1, x_2) = \lambda^{\Delta_1 + \Delta_2} f(\lambda r_{12}).
\end{equation}

\noindent To calculate the function that obeys this constraint, expand $f(r_{12}) = \sum_a f_a r_{12}^a$ in a series, noting that $a$ can be a continuous parameter, and compare to the above to get the condition that the coefficients $f_a$ are all zero, except for $a = -\Delta_1 - \Delta_2$. \\

\noindent Therefore, under translations, rotation, and dilations, we find that the Green's function is constrained to the form

\begin{equation}
G^{(2)} (x_1, x_2) = \frac{f_{-\Delta_1 - \Delta_2}}{r_{12}^{\Delta_1 + \Delta_2}} = \frac{c_{12}}{r_{12}^{\Delta_1 + \Delta_2}}
\end{equation}

\noindent Where $c_{12}$ is a constant determined by the normalization condition. \\

\noindent So far, we have come to the conclusion that the correlation function must obey a power law, and we now apply the constraints of SCTs to find that the two-point correlation function of quasi-primary fields is zero unless the two fields have the same scaling dimension $\Delta_1 = \Delta_2 = \Delta$ (\textbf{Exercise})

\begin{equation}
\bra{0} \phi_1 (x_1) \phi_2 (x_2) \ket{0} = \frac{c_{12}}{r_{12}^{2\Delta}}.
\end{equation}

\noindent This means that if we solve a system and the calculate the two-point correlation function of two fields with different scaling dimensions, $\Delta_1$ and $\Delta_2$, respectively, where $\Delta_1 \ne \Delta_2$, and get a nonzero result, then our system is not conformally invariant under the full global conformal group, but under a subgroup consisting of translations, rotations, and dilations. \\

\subsubsection*{Example: Three-Point Correlation}

\noindent Following similar procedure as in the two-point case, translations, rotations, and dilations lead to conclusion that the three-point correlation function must have the form

\begin{equation}
\bra{0} \phi_1 (x_1) \phi_2 (x_2) \phi_3 (x_3) \ket{0} = \sum_{a,b,c} \frac{c_{abc}}{r_{12}^a r_{23}^b r_{31}^c}
\end{equation}

\noindent Where the summation is constrained by the field scaling dimensions to satisfy $a+b+c = \Delta_1 + \Delta_2 + \Delta_3$. \\

\noindent Special conformal transformations further constrain the exponents in the power law to (\textbf{Exercise})

\begin{equation}
a = \Delta_1 + \Delta_2 - \Delta_3, \,\,\,\, b = -\Delta_1 + \Delta_2 + \Delta_3, \,\,\,\, c = \Delta_1 - \Delta_2 + \Delta_3.
\end{equation}

\noindent And the three-point correlation function has the form

\begin{equation}
\bra{0} \phi_1 (x_1) \phi_2 (x_2) \phi_3 (x_3) \ket{0} = \frac{c_{123}}{r_{12}^{\Delta_1 + \Delta_2 - \Delta_3} r_{23}^{-\Delta_1 + \Delta_2 + \Delta_3} r_{31}^{\Delta_1 - \Delta_2 + \Delta_3}}.
\end{equation}

\subsubsection*{Example: Four-Point Correlation}

\noindent The four-point correlation function has the form

\begin{equation}
\bra{0} \phi_1 (x_1) \phi_2 (x_2) \phi_3 (x_3) \phi_4 (x_4) \ket{0} = F \left( \frac{r_{12} r_{34}}{r_{13} r_{24}}, \frac{r_{13} r_{24}}{r_{23} r_{14}} \right) \prod_{j<k} r_{jk}^{-\Delta_j - \Delta_k + \frac{\Delta}{3}}
\end{equation}

\noindent Where $F(\cdot,\cdot)$ is an arbitrary function of the cross products and $\Delta = \sum_j \Delta_j$.

\subsection*{Conformal Theory in $(2+0)$ Dimensions}

\noindent Thus far, we have been studying the structure of conformal theories in arbitrary spacetime dimension $d$. In $(2+0)$ dimensions, conformal theories will take on a new definition. \\

\noindent Recall that we defined the complex fields $z = x_1 + i x_2$ and $\bar{z} = x_1 - i x_2$, and made the assumption that we can \textit{analytically continue} to arbitrary, independent $z$ and $\bar{z}$, such that

\begin{equation}
\Phi(x_1,x_2) = \Phi(z,\bar{z})
\end{equation}

\noindent And work out the consequences from there. We now extend our definition of quasi-primary fields, by analogy to the general, arbitrary dimension case. We will now refer to these fields as \textit{primary fields} of type, or conformal weight, $(h, \bar{h})$ with the proposed form

\begin{equation}
\Phi (z, \bar{z}) = \left( \frac{\partial f}{\partial z} \right)^h \left( \frac{\partial \bar{f}}{\partial \bar{z}} \right)^{\bar{h}} \Phi (f(z), \bar{f} (\bar{z})).
\end{equation}

\noindent Note here that there are indeed enough examples of primary fields of conformal field theories to make these assumptions interesting to study. \\

\noindent Expand the primary field infinitesimally, as it is the more convenient method opposed to working with the full global transformation, by sending $z \rightarrow z + \epsilon(z)$, and complex conjugate, to order $\epsilon$

\begin{equation}
\delta_{\epsilon, \bar{\epsilon}} \Phi (z, \bar{z}) = ((h \partial_z \epsilon(z) + \epsilon (z) \partial_z ) + (\bar{h} \partial_{\bar{z}} \bar{\epsilon}(\bar{z}) + \bar{\epsilon} (\bar{z}) \partial_{\bar{z}} )) \Phi(z, \bar{z}).
\end{equation}

\noindent And write the two-point correlation function as

\begin{equation}
G^{(2)} (\underline{z}, \bar{\underline{z}}) = \bra{0} \Phi_1 (z_1, \bar{z}_1) \Phi_2 (z_2, \bar{z}_2) \ket{0}
\end{equation}

\noindent Where $\underline{z} \equiv (z_1, z_2)$ and $\underline{\bar{z}} \equiv (\bar{z}_1, \bar{z}_2)$. \\

\noindent Applying the infinitesimal equation for $\delta_{\epsilon, \bar{\epsilon}} \Phi (z, \bar{z})$ from above, and setting equal to zero by conformal invariance to order $\epsilon$, we get

\begin{equation}
\delta_{\epsilon, \bar{\epsilon}} G^{(2)} (\underline{z}, \underline{\bar{z}}) = \bra{0} \delta_{\epsilon, \bar{\epsilon}} \Phi_1, \Phi_2 \ket{0} + \bra{0} \Phi_1, \delta_{\epsilon, \bar{\epsilon}} \Phi_2 \ket{0} = 0.
\end{equation}

\noindent From here, as before in the arbitrary dimension case, we can work out the constraints of each type of conformal transformation on the correlation function. \\

\noindent \textbf{Translation}: 

\begin{equation}
\epsilon(z) = \epsilon \implies G^{(2)}(\underline{z}, \underline{\bar{z}}) \propto z_{12} = z_1 - z_2 \text{ and }\bar{z}_{12} = \bar{z}_1 - \bar{z}_2.
\end{equation}

\noindent \textbf{Rotation \& Dilation}: 

\begin{equation}
\epsilon (z) = z \implies G^{(2)}(\underline{z}, \underline{\bar{z}}) = \frac{c_{12}}{z_{12}^{h_1 + h_2} \,\, \bar{z}_{12}^{\bar{h}_1 + \bar{h}_2}}.
\end{equation}

\noindent \textbf{SCT}: 

\begin{equation}
\epsilon (z) = z^2 \implies G^{(2)}(\underline{z}, \underline{\bar{z}}) = \frac{c_{12}}{z_{12}^{2h} \, \, \bar{z}_{12}^{2\bar{h}}}.
\end{equation}

\noindent Now, for example, suppose we have a bosonic field which applies the constraint $h-\bar{h}=0$ to the conformal weights. Setting $h+\bar{h} = \Delta$, the analytically continued two-point correlation function for the bosonic field becomes

\begin{equation}
G^{(2)}(\underline{z}, \underline{\bar{z}}) = \frac{c_{12}}{|z_{12}|^{2\Delta}}.
\end{equation}

\subsubsection*{Example: Three-Point Correlation}

\noindent Similarly, the three-point correlation function has the form

\begin{equation}
G^{(3)} (\underline{z}, \underline{\bar{z}}) = c_{123} \frac{1}{z_{12}^{h_1+h_2-h_3} z_{23}^{-h_1+h_2+h_3} z_{31}^{h_1-h_2+h_3}} \frac{1}{\bar{z}_{12}^{h_1+h_2-h_3} \bar{z}_{23}^{-h_1+h_2+h_3} \bar{z}_{31}^{h_1-h_2+h_3}}
\end{equation}

\noindent In summary, we have introduced what we consider quantum field theories and have analyzed the consequences of conformal invariance of these quantum theories. We have also analyzed subclasses of theories that obey additional constraints (e.g., $\Delta = h + \bar{h}$ is real).


\clearpage

\section*{Lecture 5: \\ Quantum CFT: Ward Identities \& Radial Quantization}
\label{sec: lec5}


\noindent We have been on this journey from classical conformal symmetries to the implementation of the symmetries in a quantum setting. Thus far, we have found that the constraints of conformal symmetry in the context of quantum field theory, in $(2+0)$, $(1+1)$, and $(d+1)$ dimensions, the two-point correlation functions must decay polynomially as

\begin{equation}
\bra{0} \hat{\phi} (x) \hat{\phi} (y) \ket{0} \propto \frac{1}{|x-y|^\alpha}.
\end{equation}

\subsection*{Radial ``Quantization''}

\noindent Note that using quotes around quantization indicates that this approach is an inspired guess, and is not derived via a functor. \\

\noindent To build a quantum field theory with conforml symmetry, invariant under the group Conf$(\mathbb{R}^{2,0})$, we need four pieces of data. \\

\noindent (1) We need a Hilbert space $\mathcal{H}$ which is preferrably an infinite, separable, kinematic space of states isomorphic to the vector space of linear operators $L^2 (\mathbb{R})$, (2) a vacuum state, or a reference vector, $\ket{0}$, a subset of linear operators in the Hilbert space $\mathcal{L}(\mathcal{H})$ called \textit{observables} $\hat{\phi} (x,y)$, $(x,y) \in \mathbb{R}^2$, which are distribution-valued objects, but can be thought of, without harm, as self-adjoint operators. \\

\noindent The choice of set of observables, the subset of linear operators from the Hilbert space, define and distinguish a quantum field theory from others. E.g., the bosonic quantum field and the hydrogen atom share the same Hilbert space, but their observables are different and define what is allowed to be measured in each quantum field theory. \\

\noindent Our conformal quantum field theory yields (projective) unitary representations of Conf$(\mathbb{R}^{2,0})$. We can use analytic continuation to map to a Minkowski theory where we have representations of Conf$(\mathbb{R}^{1,1})$. \\

\noindent Introduce the construction of ``imaginary time''

\begin{equation}
\begin{pmatrix} x \\ y \end{pmatrix} \rightarrow \begin{pmatrix} x \\ it \end{pmatrix}; \,\, t \in \mathbb{R}.
\end{equation}

\noindent If possible, the correlation functions analytically continue, such that

\begin{equation}
\bra{0} \hat{\phi} (x,y) \hat{\phi} (0,0) \ket{0} = f(x,y) \rightarrow f(x,it).
\end{equation}

\noindent If all $n$-point correlation functions are analytically continued in this way, it \textit{often} happens that we get a mapping

\begin{equation}
G^{(n)} ((x_1,y_1), \dots) \rightarrow G^{(n)} ((x_1,t_1), \dots)
\end{equation}

\noindent Which are invariant, respectively, under symmetry transformations

\begin{equation}
\text{Conf}(\mathbb{R}^{2,0}) \rightarrow \text{Conf}(\mathbb{R}^{1,1}).
\end{equation}

\noindent This is the analytic continuation to a \textit{Minkowski theory}, and the criteria to ensure that this mapping exists is called \textit{reflection positivity}. See the work of Glimm and Jaffe for a rigorous account of this machinery. \\

\noindent With a Minkowski quantum field theory, enter the Heisenberg picture to apply time evolution, and analytically continue the ``time'' variable to define ``inverse temperature'' $t \rightarrow -i\beta$, which essentially gives the correlation function for the thermal state of the Hamiltonian $\hat{H}$ with inverse temperature $\beta$

\begin{align}
\bra{0} \hat{\phi} (x,y) \hat{\phi} (0,0) \ket{0} &= \bra{0} e^{-it \hat{H}} \hat{\phi} (x,0) e^{it\hat{H}} \hat{\phi} (0,0) \ket{0} \\
&= \bra{0} e^{-\beta \hat{H}} \hat{\phi} (x,0) e^{\beta \hat{H}} \hat{\phi} (0,0) \ket{0}.
\end{align}

\noindent \textbf{Digression}: What does ``physical'' mean for this course? \\

\noindent To define a physical theory, we require \textit{kinematics} and \textit{observables}, which make for a perfectly fine physical theory, even without \textit{dynamics}. \\

\noindent The kinematics of a system introduces a Hilbert space $\mathcal{H}$ and a set of density operators $\rho(\mathcal{H})$ that describe the states of the system, but do not allow measurement of those states. The observables of a system $\mathcal{O} \subset \mathcal{L}(\mathcal{H})$ are the measurements of the states of the system, and are labelled by points in the underlying manifold $\mathcal{M}$ of the theory, such that $x, x^\dagger, \mathbb{I} \in \mathcal{O}, \, \text{ for } x \in \mathcal{M}$. \\

\noindent Dynamics of a system are introduced via the group of isometries on the manifold $G = \text{Isom}(\mathcal{M})$, and, for a quantum field theory, we search for projective, unitary representations of $G$ on the Hilbert space. The construct of time only comes in as a choice of the manifold and its one-dimensional subgroups that act like time, but time is not necessarily an axiom of a quantum theory. \\

\noindent \textbf{End Digression}. \\

\noindent So far, we have labelled our observables in our Minkowski CFT by $(\sigma^0, \sigma^1) \in \mathbb{R}^{1,1}$. We now complexify these coordinates by ``Euclideanizing'' and sending the timelike coordinate to be imaginary, such that $\sigma^0 \rightarrow i \sigma^0$, and 

\begin{equation}
z = \sigma^1 + i \sigma^0 \text{ and } \bar{z} = \sigma^1 - i \sigma^0
\end{equation}

\noindent Which correspsonds to an analytic continuation of the $n$-point correlation functions

\begin{equation}
G^{(n)} (\underline{\sigma}_1, \underline{\sigma}_2, \dots) \rightarrow G^{(n)} (\underline{z}_1, \bar{\underline{z}}_1; \underline{z}_2, \bar{\underline{z}}_2; \dots).
\end{equation}

\noindent In this complexifed, Euclidean spacetime, with locations defined by coordinates $z$ and $\bar{z}$, compactify space to a cylinder, such that space corresponds to the transverse direction $\sigma^1 \rightarrow \sigma^1 + 2\pi$, and time corresponds to the longitudinal direction, and our complex coordinates are now

\begin{equation}
z = e^{\sigma^1 + i \sigma^0} \text{ and } \bar{z} = e^{\sigma^1 - i \sigma^0}.
\end{equation}

\noindent Circles in the complex plane correspond to constant time, the radial direction on the cylinder, and the real coordinates $\sigma^0, \sigma^1$ are mapped to the complex coordinates as

\begin{equation}
\sigma^0 = \infty \rightarrow z = \infty \text{ and  } \sigma^0 = -\infty \rightarrow z=0
\end{equation}

\noindent And transformations maps as

\begin{align}
\text{Time reversal } [\sigma^0 \rightarrow -\sigma^0] &\rightarrow \text{Inversion } [z \rightarrow \frac{1}{z}] \\
\text{Time translation } [\sigma^0 \rightarrow \sigma^0 + a] &\rightarrow \text{Dilation } [z \rightarrow e^a z] \\
\text{Spatial translation } [\sigma^1 \rightarrow \sigma^1 + a] &\rightarrow \text{Rotation } [z \rightarrow e^{ia} z] 
\end{align}

\begin{figure}[H]
	\centering
	\includegraphics[width=3.5in]{images/radial_quantization.png} 
\end{figure} 

\noindent Now we make a move to constructing a conformal field theory in complexified, Euclidean spacetime. \\

\subsection*{Ward Identities}

\noindent Witha conformally invariant classical theory, we work towards getting a collection of observables obeying the correct symmetry for a conformally invariant quantum theory. \\

\noindent Recall from Noether's theorem that conserved charges correspond to classical symmetries $Q$ obeying the anticommutation bracket

\begin{equation}
\{ Q_j, Q_k \} = f_{jk}^{\,\,\,\,l} Q_l.
\end{equation}

\noindent We may try to naively quantize by ``puttings hats on'' the conserved quantities that obey the commutation brackets

\begin{equation}
[ \hat{Q}_j, \hat{Q}_k ] = i f_{jk}^{\,\,\,\,l} \hat{Q}_l,
\end{equation}

\noindent But without a Hilbert space, we can not write these operators as functions of the fields (e.g., $\hat{Q} = f(\hat{\phi})$. \\

\noindent From the classical symmetries, we now derive the quantum generators of the symmetries. \\

\noindent Suppose we have a classical field theory with action $S[\underline{\phi}]$, where $\underline{\phi}$ is a vector of classical fields. Assume that $S$ is symmetric under the Lie group of infinitesimal transformations, such that

\begin{equation}
\underline{\phi}' (\underline{x}) = \underline{\phi} (\underline{x}) - i \omega_a (\underline{x}) \textbf{G}_a \underline{\phi} (\underline{x}) = e^{-i \omega_a (\underline{x}) \textbf{G}_a} \underline{\phi} (\underline{x})
\end{equation}

\noindent Where $\omega_a (\underline{x})$ is infinitesimal and $\textbf{G}_a$ is a matrix acting on the vector labels of $\underline{\phi}$. \\

\noindent Use the path integral prescription to work out how this transformation affects the correlation functions

\begin{equation}
\bra{0} \underline{\hat{\phi}} (\underline{x}_1) \dots \underline{\hat{\phi}} (\underline{x}_n) \ket{0} = \lim_{T \rightarrow \infty(1-i\epsilon)} \frac{\int \mathcal{D}\phi \, \phi (\underline{x}_1) \dots \phi (\underline{x}_n) e^{iS[\underline{\phi}]}}{\int \mathcal{D} \phi \, e^{i S[\underline{\phi}]}}.
\end{equation}

\noindent The first step is to change variables $\underline{\phi} \rightarrow \underline{\phi}'$ and assume that the measure is invariant $\mathcal{D}\phi' = \mathcal{D}\phi$. Define the operator $\hat{X}$ to be the time-ordered product of quantum fields

\begin{equation}
\hat{X} = \mathcal{T} [ \underline{\hat{\phi}} (\underline{x}_1) \dots \underline{\hat{\phi}} (\underline{x}_n)].
\end{equation}

\noindent Note that the time-ordering will be assumed from here on, and consider the expectation value of $\hat{X}$ in the path integral prescription

\begin{equation}
\langle \hat{X} \rangle = \frac{\int \mathcal{D} \phi' \, \left( \underline{\hat{\phi}} (\underline{x}_1) \dots \underline{\hat{\phi}} (\underline{x}_n) + i \omega_a (\underline{x}) \textbf{G}_a (\underline{\hat{\phi}} (\underline{x}_1) \dots \underline{\hat{\phi}} (\underline{x}_n) + \dots \right) e^{i (S + \int dx \, \partial_\mu j_a^\mu \omega_a (\underline{x}))}}{\int \mathcal{D}\phi \, e^{iS}}.
\end{equation}

\noindent To zeroth order in $\omega_a$, this is exactly the correlation function prior to the change of variables. To first order in $\omega_a$, we get the equality for quantum fields

\begin{equation}
\frac{\partial}{\partial x^\mu} \langle \hat{j}_a^\mu (\underline{x}) \hat{\phi} (\underline{x}_1) \dots \hat{\phi} (\underline{x}_n) \rangle = -i \sum_{j=1}^{n} \delta(\underline{x} - \underline{x}_j) \langle \hat{\phi} (\underline{x}_1) \dots \textbf{G}_a \hat{\phi} \left( \underline{x}_j) \dots \hat{\phi} (\underline{x}_n) \right)\rangle.
\end{equation}

\noindent The generator of symmetry $\hat{j}_a^\mu (\underline{x})$ is handed over by the oath integral approach.

\clearpage

\section*{Lecture 6: Ward Identities}
\label{sec: lec6}

\noindent We are on our way to building a conformal quantum field theory (CQFT). So far, we have analyzed the consequences of conformal symmetries on the observables of a quantum field theory. \\

\noindent The observables of a quantum field theory, not necessarily self-adjoint as in quantum theory, are indirectly observable, as they are the correlation functions derivable from the direct observables of scattering amplitudes in scattering experiments. \\



%
\noindent We are on our way to building a conformal quantum field theory (CQFT). So far, we have analyzed the consequences of conformal symmetries on the observables of a quantum field theory. \\

\noindent The observables of a quantum field theory, not necessarily self-adjoint as in quantum theory, are indirectly observable, as they are the correlation functions derivable from the direct observables of scattering amplitudes in scattering experiments. \\

\noindent We have found that the two-point correlation functions under the assumed constraints of a conformal field theory behave as

\begin{equation}
\langle \hat{\phi}_\alpha (x) \hat{\phi}_\beta (y) \rangle \sim \frac{1}{|x-y|^{d_\alpha + d_\beta}}.
\end{equation}

\noindent To actually construct a conformal quantum field theory (CQFT), we use the path integral approach as an efficient tool to (hopefully) yield proper, according to the theory at hand, quantum observables. In the path integral sppraoch, we have a ``box'' into which we put classical data, namely the action $S[\phi]$, a functional of the fields, and get wuantum data as output in the form of time-ordered correlation functions

\begin{equation}
\langle \mathcal{T} [ \hat{\phi} (x_1) \dots \hat{\phi} (x_n) ] \rangle \equiv \frac{\int \mathcal{D}\phi \, \phi (x_1) \dots \phi (x_n) e^{iS[\phi]}}{\int \mathcal{D} \phi \, e^{iS[\phi]}}.
\end{equation}

\noindent \textbf{Goal}: Our goal is to take a classical CFT and calculate the quantum generators of conformal symmetries corresponding to some CQFT. \\

\noindent \textbf{Note on imaginary time}: \\

\noindent We work with imaginary time $t \rightarrow -i \beta$ via analytic continuation, also called Wick rotation for two reasons.

\begin{enumerate}
\item It is convenient since things will converge better.
\item It makes direct contact with statistical physics.
\end{enumerate}

\noindent To the second point, Wick rotation of the time coordinate turns all formulae for a quantum system, eith unitary time evolution, into statements about partition functions and thermal systems at temperature $\beta$. In other words, Wick rotation maps the unitary generators of time translation to a non-unitary semigroup with fixed points being the ground state

\begin{equation}
U(t) = e^{-it\hat{H}} \rightarrow S(\beta) = e^{-\beta \hat{H}}.
\end{equation}

\noindent From $S(\beta)$, we simply take the trace and get the partition function, for a system in thermal equilibrium with a reservoir at inverse temperature $\beta$

\begin{equation}
Z = \text{tr} (S(\beta)) = \text{tr} (e^{-\beta \hat{H}}).
\end{equation}

\noindent This contact with statistical physics and thermal states of systems is also interesting because in studying phase transitions, at critical points, the symmetry group is enlarged, and the system exhibits, for example, dilation and scale invariance. Phase transitions are examples of conformal field theories, though with non-unitary representations of the symmetry group. See \textit{Cardy, Scaling \& Phase Transitions} for reference. \\

\noindent Since our goal is to construct unitary representations of the conformal group of Minkowski space, for the reasons above, we Wick rotate and work with imaginary time. After our calculations, we Wick rotate back into real time, as much as allowed by holomorphic functions. \\

\noindent The path integral, our tool for calculating quantum obervables, under Wick rotation $t \rightarrow -i \beta$ transforms as 

\begin{equation}
\int \mathcal{D} \phi \, e^{i S[\phi]} \rightarrow \int \mathcal{D} \phi \, e^{iS[\phi]}.
\end{equation}

\subsection*{Constructing a CQFT}

\noindent Suppose we have a tuple of classical fields $\phi(x)$, and let $S$ be the action of the fields with assumed invariance under infinitesimal conformal symmetry transformations (up to a total derivative).

\begin{equation}
\phi(x) \rightarrow \phi' (x) = \phi (x) - i \omega_a (x) G_a \phi (x)
\end{equation}

\noindent Where $G_a$ is a symmetry transformation matrix. For example,

\begin{align}
\text{Translations: }& G_a \simeq -i \partial_\nu \\
\text{Dilations: }& G_a \simeq -i x^\nu \partial_\nu - i \Delta_a.
\end{align}

\noindent Note that time-ordering of quantum field operators is implicit whenever the LHS is a correlator and the RHS is a path integral, and define 

\begin{equation}
\hat{X} \equiv \mathcal{T} [ \hat{\phi} (x_1) \dots \hat{\phi} (x_n) ].
\end{equation}

\noindent By the path integral prescription, the time-ordered quantum correlation function is given by classical (not time-ordered) data, after Wick rotation, as

\begin{equation}
\langle \hat{X} \rangle = \frac{1}{Z} \int \mathcal{D} \phi \, X e^{iS[\phi]} = \frac{1}{Z} \int \mathcal{D} \phi \, \phi (x_1) \dots \phi (x_n) e^{iS[\phi]}
\end{equation}

\noindent Where we have written the partition function $Z$, since we have Wick-rotated into imaginary time and

\begin{equation}
\int \mathcal{D} \phi \, e^{iS} \rightarrow \text{tr}(e^{-\beta H}) = Z.
\end{equation}

\noindent Under the infinitesimal transformation above, \textit{assume} that the measure is invariant, such that $\mathcal{D} \phi' = \mathcal{D} \phi$. Recall that the action is invariant up to a total derivative. Then the correlation function, including the product of time-ordered quantum field operators, becomes

\begin{equation}
\langle \hat{X} \rangle = \frac{1}{Z} \int \mathcal{D} \phi' \,\, (X + \delta X) e^{-S[\phi] - \int dx \, \partial_\mu j_a^\mu \omega_a (x)}
\end{equation}

\noindent Where $X' = X + \delta X$, and to order $\omega_a$

\begin{equation}
X' = (\phi (x_1) - i \omega_a G_a \phi (x_1)) (\phi (x_2) - i \omega_a G_a \phi (x_2)) \dots
\end{equation}

\noindent Which makes the infinitesimal change in $X$ to be

\begin{align}
\delta X &= -i \sum_{j=1}^n \, \left(\phi(x_1) \dots G_a \phi(x_j) \dots \phi(x_n) \right) \omega_a (x_j) \\
&= -i \int dx \,\, \omega_a (x) \sum_{j=1}^n  \, \left(\phi(x_1) \dots G_a \phi(x_j) \dots \phi(x_n) \right) \delta (x-x_j).
\end{align}

\noindent Insert this into the equation for $\langle \hat{X} \rangle$ and distribute to get

\begin{align}
\langle \hat{X} \rangle &= \frac{1}{Z} \int \mathcal{D} \phi \left( X e^{-S} + \delta X e^{-S} + X e^{-S}\left(\int dx \partial_\mu j_a^\mu \omega_a (x)\right) \right) \\
\langle \hat{X} \rangle &= \langle \hat{X} \rangle + \langle \delta \hat{X} \rangle - \int dx \, \partial_\mu \langle \hat{j}_a^\mu \hat{X} \rangle \omega_a (x) \\
0 &= \langle \delta \hat{X} \rangle - \int dx \, \partial_\mu \langle \hat{j}_a^\mu \hat{X} \rangle \omega_a (x).
\end{align}

\noindent By the infintesimal transformation and the invariance of the action, we have inserted the conserved classical current into the path integral presecription and have gotten a quantum operator in return! We find that the quantization of the infinitesimal change $\delta X$ is equal to the integral of the divergence of the quantization of the product $j_a^\mu X$. \\

\noindent From this equality, which holds for all $\omega_a (x)$, extract the local relation of how the quantum fields transform according to the transformation-generating conserved current. Pull off the integral by the insertion of the delta function into $\delta X$ above. The local equality, called the \textit{Ward identity} is then

\begin{equation}
\frac{\partial}{\partial x^\mu} \langle \hat{j}_a^\mu (x) \hat{\phi} (x_1) \dots \hat{\phi} (x_n) \rangle = -i \sum_{j=1}^n \, \delta (x-x_j) \langle\hat{\phi} (x_1) \dots G_a \hat{\phi} (x_j) \dots \hat{\phi} (x_n) \rangle
\end{equation}

\noindent The Ward identity is the principal tool by which we quantize symmetries and showing how the symmetries are implemented on quantum correlation functions. \\

\noindent Integrate the Ward identity over all of spacetime (assuming good decay), and show that the quantum correlation function is invariant under the infinitesimal symmetry transformation

\begin{equation}
\delta_\omega \langle \hat{\phi} (x_1) \dots \hat{\phi} (x_n) \rangle \equiv -i \omega_a (x) \sum_{j=1}^n \, \langle\hat{\phi} (x_1) \dots G_a \hat{\phi} (x_j) \dots \hat{\phi} (x_n) \rangle = 0.
\end{equation}

\noindent Recall that the conserved charge corresponding to the \textit{generator} of some symmetry is gotten by integrating the time component of the conserved current

\begin{equation}
\hat{Q}_a = \int d^{d_a} x \,\, \hat{j}_a^0 (x).
\end{equation}

\noindent Integrate the Ward identity now over a thin slice of time $t_- < t < t_+$, instead of the entire spacetime, and suppose that there is one point of time $x_1^0 = t$ that is contained within the slice, and all others are sufficiently far away.

\begin{figure}[H]
	\centering
	\includegraphics[width=2in]{images/ward_timeslice.png} 
\end{figure} 

\noindent The LHS of the Ward identity becomes 

\begin{equation}
\langle \hat{Q}_a (t_+) \hat{\phi} (x_1) \hat{Y} \rangle - \langle \hat{Q}_a (t_-) \hat{\phi} (x_1) \hat{Y} \rangle = -i \langle G_a \hat{\phi} (x_1) \hat{Y} \rangle
\end{equation}

\noindent Where $\hat{Y} = \hat{\phi} (x_2) \dots \hat{\phi} (x_n)$. \\

\noindent Let the inverse temperature go to infinity $\beta \rightarrow \infty$, such that we have the vacuum correlation function $\langle \hat{X} \rightarrow \bra{0} \hat{X} \ket{0}$. Suppose that there is a point of time ahead of the slice-contained point $x_j^0 > x_1^0$, and in the limit of an inifnitesimal time slice$t_- \rightarrow t_+$, the expression for the Ward identity above becomes

\begin{equation}
\bra{0} [ \hat{Q}_a, \hat{\phi} (x_1) ] \hat{Y} \ket{0} = -i \bra{0} G_a \hat{\phi} (x_1) \ket{0}.
\end{equation}

\noindent This is true for all $\hat{Y}$, such that we have the commutator bracket, starting with classical data, for conserved charges and quantum field operators that shows that passing the quantum conserved charge operator over the field is tantamount to applying the symmetry transformation $G_a$

\begin{equation}
[ \hat{Q}_a, \hat{\phi} ] = -i G_a \hat{\phi}.
\end{equation}

\noindent For example, translations transform the field as $\phi' (x) = \phi (x) - \epsilon^\mu \partial_\mu \phi (x)$ with symmetry transformation matrix $G_a = i \partial_\mu$, and the conserved charge operator is the total momentum operator $\hat{Q}_\mu \equiv \hat{P}_\mu$. \\

\subsection*{Ward Identities for Conformal Symmetries}

\noindent Let $\hat{X}$ be the product of local quantum field operators as before, and consider the invariants of conformal symmetry transformations. The conserved current is the \textit{energy-momentum tensor} $\hat{T}^\mu_\nu$, where we assume that $T_\mu^\nu$ (classical) is traceless $T^\mu_\mu = 0$ and symmetric $T_{\mu\nu} = T_{\nu\mu}$, which does not necessarily hold quantumly. \\

\noindent \textbf{Translation}: \\

\begin{equation}
\partial_\mu \langle \hat{T}^\mu_\nu \hat{X} \rangle = -i \sum_j \delta(x-x_j) \partial_{x_j^\nu} \langle \hat{X} \rangle.
\end{equation}

\noindent \textbf{Rotation}: \\

\begin{equation}
\langle (\hat{T}^{\rho\nu} - \hat{T}^{\nu\rho} ) \hat{X} \rangle = -i \sum_j \delta(x-x_j) s_j^{\nu\rho} \langle \hat{X} \rangle
\end{equation}

\noindent Since $j^{\mu\nu\rho} = T^{\mu\nu} x^\rho - T^{\mu\rho} x^\nu$ (\textbf{Exercise}). Note that $s_j^{\nu\rho}$ is the spin matrix representation of the $j^{th}$ field vector, and is equal to one for spinless particles. \\

\noindent \textbf{Dilation}: \\

\begin{equation}
\langle \hat{T}^\mu_\mu \hat{X} \rangle = - \sum_j \delta(x-x_j) \Delta_j \langle \hat{X} \rangle.
\end{equation}

\noindent Since $G \equiv D = -i x^\nu \partial_\nu - i \Delta$ (\textbf{Exercise}). \\ \\

\noindent Next, we work out the two-dimensional case and rewrite the Ward identities in complex coordinates. We will then work out examples of energy-momentum tensors for free bosons, and see that the remaining degrees of freedom of the energy-momentum tensor for the \textit{Virasoro algebra}.

%\clearpage

\end{document}